\documentclass[12pt, titlepage]{article}

\usepackage{amsmath, mathtools}

\usepackage[round]{natbib}
\usepackage{amsfonts}
\usepackage{amssymb}
\usepackage{graphicx}
\usepackage{colortbl}
\usepackage{xr}
\usepackage{hyperref}
\usepackage{longtable}
\usepackage{xfrac}
\usepackage{tabularx}
\usepackage{float}
\usepackage{siunitx}
\usepackage{booktabs}
\usepackage{multirow}
\usepackage[section]{placeins}
\usepackage{caption}
\usepackage{fullpage}
\usepackage{ulem}
\hypersetup{
    colorlinks,
    citecolor=blue,
    filecolor=black,
    linkcolor=red,
    urlcolor=blue
}

\input{../../Comments}
%% Common Parts

\newcommand{\progname}{SE 4G06, TRON 4TB6}
\newcommand{\authname}{Team 26, STRONE 
\\ Jordan Bierbrier
\\ Azriel Gingoyon
\\ Taranjit Lotey
\\ Udeep Shah
\\ Abraham Taha
}                  

\usepackage{hyperref}
    \hypersetup{colorlinks=true, linkcolor=blue, citecolor=blue, filecolor=blue,
                urlcolor=blue, unicode=false}
    \urlstyle{same}
                                


\newcounter{acnum}
\newcommand{\actheacnum}{AC\theacnum}
\newcommand{\acref}[1]{AC\ref{#1}}

\newcounter{ucnum}
\newcommand{\uctheucnum}{UC\theucnum}
\newcommand{\uref}[1]{UC\ref{#1}}

\newcounter{mnum}
\newcommand{\mthemnum}{M\themnum}
\newcommand{\mref}[1]{M\ref{#1}}

\begin{document}

\title{Module Guide for \progname{}} 
\author{\authname}
\date{\today}

\maketitle

\pagenumbering{roman}

\section*{Revision History}

\begin{tabularx}{\textwidth}{p{3cm}p{2cm}X}
\toprule {\bf Date} & {\bf Version} & {\bf Notes}\\
\midrule
1/17/2023 & 1.0 & Added Content\\
1/18/2023 & 1.1 & Added Graph\\
4/5/2023 & 1.2 & Updated for feedback and consistency\\
\bottomrule
\end{tabularx}

\newpage

\tableofcontents

\listoffigures

\listoftables

\newpage

\section*{Reference Material}

This section records information for easy reference.

\subsection*{Abbreviations and Acronyms}

\renewcommand{\arraystretch}{1.2}
\begin{tabular}{l l} 
  \toprule		
  \textbf{symbol} & \textbf{description}\\
  \midrule 
  AC & Anticipated Change\\
  DAG & Directed Acyclic Graph \\
  M & Module \\
  MG & Module Guide \\
  OS & Operating System \\
  R & Requirement\\
  SC & Scientific Computing \\
  SRS & Software Requirements Specification\\
  \progname & Explanation of program name\\
  UC & Unlikely Change \\
  \bottomrule
\end{tabular}\\

\newpage

\pagenumbering{arabic}

\section{Introduction}

Decomposing a system into modules is a commonly accepted approach to developing
software.  A module is a work assignment for a programmer or programming
team~\citep{ParnasEtAl1984}.  We advocate a decomposition
based on the principle of information hiding~\citep{Parnas1972a}.  This
principle supports design for change because the ``secrets'' that each module
hides represent likely future changes.  Design for change is valuable in SC,
where modifications are frequent, especially during initial development as the
solution space is explored.  

\noindent Our design follows the rules layed out by \citet{ParnasEtAl1984}, as follows:
\begin{itemize}
\item System details that are likely to change independently should be the
  secrets of separate modules.
\item Each data structure is implemented in only one module.
\item Any other program that requires information stored in a module's data
  structures must obtain it by calling access programs belonging to that module.
\end{itemize}

\noindent After completing the first stage of the design \textcolor{red}{and} the Software Requirements
Specification (SRS), the Module Guide (MG) is developed~\citep{ParnasEtAl1984}. The MG
specifies the modular structure of the system and is intended to allow both
designers and maintainers to easily identify the parts of the software.  The
potential readers of this document are as follows:

\begin{itemize}
\item New project members: This document can be a guide for a new project member
  to easily understand the overall structure and quickly find the
  relevant modules they are searching for.
\item Maintainers: The hierarchical structure of the module guide improves the
  maintainers' understanding when they need to make changes to the system. It is
  important for a maintainer to update the relevant sections of the document
  after changes have been made.
\item Designers: Once the module guide has been written, it can be used to
  check for consistency, feasibility, and flexibility. Designers can verify the
  system in various ways, such as consistency among modules, feasibility of the
  decomposition, and flexibility of the design.
\end{itemize}

\noindent The rest of the document is organized as follows. Section
\ref{SecChange} lists the anticipated and unlikely changes of the software
requirements. Section \ref{SecMH} summarizes the module decomposition that
was constructed according to the likely changes. Section \ref{SecConnection}
specifies the connections between the software requirements and the
modules. Section \ref{SecMD} gives a detailed description of the
modules. Section \ref{SecTM} includes two traceability matrices. One checks
the completeness of the design against the requirements provided in the SRS. The
other shows the relation between anticipated changes and the modules. Section
\ref{SecUse} describes the use relation between modules.

\section{Anticipated and Unlikely Changes} \label{SecChange}

This section lists possible changes to the system. According to the likeliness
of the change, the possible changes are classified into two
categories. Anticipated changes are listed in Section \ref{SecAchange}, and
unlikely changes are listed in Section \ref{SecUchange}.

\subsection{Anticipated Changes} \label{SecAchange}

Anticipated changes are the source of the information that is to be hidden
inside the modules. Ideally, changing one of the anticipated changes will only
require changing the one module that hides the associated decision. The approach
adapted here is called design for
change.

\begin{description}
\item[\refstepcounter{acnum} \actheacnum \label{ac1}:] Upgraded microcontroller to higher specifications (i.e. more RAM, better processor) will require revisions of the code.
\item[\refstepcounter{acnum} \actheacnum \label{ac2}:] Sound recognition algorithm will likely improve / be more efficient over time (better algorithm).
\item[\refstepcounter{acnum} \actheacnum \label{ac3}:] Code will be modified to run on IOS.
\item[\refstepcounter{acnum} \sout{\actheacnum} \label{ac4}:] \sout{More efficient battery.} \textcolor{red}{(Out of Scope)}
\item[\refstepcounter{acnum} \actheacnum \label{ac5}:] Bluetooth handler may change with newer bluetooth implementations are adopted.
\item[\refstepcounter{acnum} \actheacnum \label{ac6}:] Code adjusted for higher resolution microphone.
\item[\refstepcounter{acnum} \sout{\actheacnum} \label{ac7}:] \sout{Upgrading encryption algorithm for user login information.} \textcolor{red}{(Design Decision was made to remove project's authentication aspects)}
\end{description}

\subsection{Unlikely Changes} \label{SecUchange}

The module design should be as general as possible. However, a general system is
more complex. Sometimes this complexity is not necessary. Fixing some design
decisions at the system architecture stage can simplify the software design. If
these decision should later need to be changed, then many parts of the design
will potentially need to be modified. Hence, it is not intended that these
decisions will be changed.

\begin{description}
  \item[\refstepcounter{ucnum} \uctheucnum \label{uc1}:] Input data i.e. sound that is processed by software.
  \item[\refstepcounter{ucnum} \uctheucnum \label{uc2}:] Output data being in the form of haptic feedback.
  \item[\refstepcounter{ucnum} \uctheucnum \label{uc3}:] Bluetooth being the sole communication method between phone application and microcontroller code.
  \item[\refstepcounter{ucnum} \uctheucnum \label{uc4}:] Signal processing done on microcontroller as opposed to phone.
\end{description}

\section{Module Hierarchy} \label{SecMH}

This section provides an overview of the module design. Modules are summarized
in a hierarchy decomposed by secrets in Table \ref{TblMH}. The modules listed
below, which are leaves in the hierarchy tree, are the modules that will
actually be implemented.

\begin{description}
\item [\refstepcounter{mnum} \mthemnum \label{mHH}:] Hardware-Hiding Module
\item [\refstepcounter{mnum} \sout{\mthemnum} \label{mLogin}:] \sout{Login Module} \textcolor{red}{(Design Decision was made to remove project's authentication aspects)}
\item [\refstepcounter{mnum} \sout{\mthemnum} \textcolor{red}{M2} \label{mBluetooth}:] Bluetooth Connection Module
\item [\refstepcounter{mnum} \sout{\mthemnum} \textcolor{red}{M3} \label{mKeyword}:] Keyword Selection Module
\item [\refstepcounter{mnum} \sout{\mthemnum} \textcolor{red}{M4} \label{mOutput}:] Output Signal Module
\item [\refstepcounter{mnum} \sout{\mthemnum} \label{mProfile}:] \sout{Profile Module}
\item [\refstepcounter{mnum} \sout{\mthemnum} \label{mBattery}:] \sout{Battery Status Module} \textcolor{red}{(Out of Scope)}
\item [\refstepcounter{mnum} \sout{\mthemnum} \textcolor{red}{M5} \label{mSound}:] Sound Classification Module
\item [\refstepcounter{mnum} \sout{\mthemnum} \textcolor{red}{M6} \label{mComm}:] Bluetooth Communication Module
\item [\refstepcounter{mnum} \sout{\mthemnum} \textcolor{red}{M7} \label{mMicrophone}:] Microphone Module
\end{description}


\begin{table}[h!]
\centering
\begin{tabular}{p{0.3\textwidth} p{0.6\textwidth}}
\toprule
\textbf{Level 1} & \textbf{Level 2}\\
\midrule
{Hardware-Hiding Module \textcolor{red}{(M1)}} & ~ \\
\midrule

\multirow{7}{0.3\textwidth}{Behaviour-Hiding Module} & \sout{Login Module} \textcolor{red}{(Design Decision - Removed)}\\
& Bluetooth connection Module \textcolor{red}{(M2)}\\
& Keyword Selection Module \textcolor{red}{(M3)}\\
& Output Signal Module \textcolor{red}{(M4)}\\
& \sout{Profile Module}\\ 
& \sout{Battery Status Module} \textcolor{red}{(Out of Scope)}\\
\midrule

\multirow{3}{0.3\textwidth}{Software Decision Module} & {Sound Classification Module} \textcolor{red}{(M5)}\\
& Bluetooth Communication Module \textcolor{red}{(M6)}\\
& Microphone Module \textcolor{red}{(M7)}\\
\bottomrule

\end{tabular}
\caption{Module Hierarchy}
\label{TblMH}
\end{table}

\section{Connection Between Requirements and Design} \label{SecConnection}

The design of the system is intended to satisfy the requirements developed in
the SRS. In this stage, the system is decomposed into modules. The connection
between requirements and modules is listed in Table~\ref{TblRT}.

\section{Module Decomposition} \label{SecMD}

Modules are decomposed according to the principle of ``information hiding''
proposed by \citet{ParnasEtAl1984}. The \emph{Secrets} field in a module
decomposition is a brief statement of the design decision hidden by the
module. The \emph{Services} field specifies \emph{what} the module will do
without documenting \emph{how} to do it. For each module, a suggestion for the
implementing software is given under the \emph{Implemented By} title. If the
entry is \emph{OS}, this means that the module is provided by the operating
system or by standard programming language libraries.  \emph{\progname{}} means the
module will be implemented by the \progname{} software.

Only the leaf modules in the hierarchy have to be implemented. If a dash
(\emph{--}) is shown, this means that the module is not a leaf and will not have
to be implemented.

\subsection{Hardware Hiding Modules (\mref{mHH})}

\begin{description}
\item[Secrets:]The data structure and algorithm used to implement the virtual
  hardware.
\item[Services:]Serves as a virtual hardware used by the rest of the
  system. This module provides the interface between the hardware and the
  software. So, the system can use it to display outputs or to accept inputs.
\item[Implemented By:] Application OS
\end{description}

\subsection{Behaviour-Hiding Module}

\begin{description}
\item[Secrets:]The contents of the required behaviours.
\item[Services:]Includes programs that provide externally visible behaviour of
  the system as specified in the software requirements specification (SRS)
  documents. This module serves as a communication layer between the
  hardware-hiding module and the software decision module. The programs in this
  module will need to change if there are changes in the SRS.
\item[Implemented By:] --
\end{description}

\subsubsection{\sout{Login Module (\mref{mLogin})} \textcolor{red}{Design Decision - Removed}}
\begin{description}
\item[Secrets:] Login authentication algorithm and profile data. 
\item[Services:] Login capabilities into the application.
\item[Implemented By:] Synesthesia Wear
\end{description}

\subsubsection{Bluetooth Connection Module \sout{(\mref{mBluetooth})} \textcolor{red}{(M2)}}
\begin{description}
  \item[Secrets:] How the Bluetooth connection is established with the application.
  \item[Services:] Module is used to connect the application to the microcontroller on the wearable device. User clicks the pairing button and the application establishes a connection with an external device.
  \item[Implemented By:] Arduino Library and Android Studio Library
  \end{description}

\subsubsection{Keyword Selection Module \sout{(\mref{mKeyword})} \textcolor{red}{(M3)}}
\begin{description}
  \item[Secrets:] How the user’s keyword is communicated to the microcontroller/device and their profile data is updated.
  \item[Services:] Provides user with option of keyword selection and corresponding custom haptic feedback.
  \item[Implemented By:] Synesthesia Wear
  \end{description}

\subsubsection{Output Signal Module \sout{(\mref{mOutput})} \textcolor{red}{(M4)}}
\begin{description}
  \item[Secrets:] How and when the microcontroller communicates with the vibration motor.
  \item[Services:] Provide haptic feedback to the user through the use of a vibration motor.
  \item[Implemented By:] Synesthesia Wear and Arduino Library
  \end{description}

\subsubsection{\sout{Profile Module (\mref{mProfile})}}
\begin{description}
  \item[Secrets:] The database structure which contains the data for the users.
  \item[Services:] Allows users to login, logout, and update profile settings (keywords, age, name, date of registration).
  \item[Implemented By:] Synesthesia Wear
  \end{description}

\subsubsection{\sout{Battery Status Module (\mref{mBattery})} \textcolor{red}{(Out of Scope)}} 
\begin{description}
  \item[Secrets:] Algorithm using voltage measurements to estimate the remaining battery life.
  \item[Services:] Displays remaining battery life of wearable device on the user application. Additionally, alerts user on application when battery below 10\% remaining life.
  \item[Implemented By:] Synesthesia Wear
  \end{description}


\subsection{Software Decision Module}

\begin{description}
\item[Secrets:] The design decision based on mathematical theorems, physical
  facts, or programming considerations. The secrets of this module are
  \emph{not} described in the SRS.
\item[Services:] Includes data structure and algorithms used in the system that
  do not provide direct interaction with the user. 
  % Changes in these modules are more likely to be motivated by a desire to
  % improve performance than by externally imposed changes.
\item[Implemented By:] --
\end{description}

\subsubsection{Sound Classification Module \sout{(\mref{mSound})} \textcolor{red}{(M5)}}
\begin{description}
  \item[Secrets:] Algorithm used for sound classification. Additionally, filters out sound below specified intensity.
  \item[Services:] Provides output to vibration motor if keyword is classified.
  \item[Implemented By:] Synesthesia Wear and Arduino Library
  \end{description}

\subsubsection{Bluetooth Communication Module \sout{(\mref{mComm})} \textcolor{red}{(M6)}}
\begin{description}
  \item[Secrets:] Communication protocol using preexisting connection to communicate between application and microcontroller.
  \item[Services:] Updates microcontroller with selected keywords, feedback preferences, and battery voltage.
  \item[Implemented By:] Arduino Library and Android Studio Library
  \end{description}

\subsubsection{Microphone Module \sout{(\mref{mMicrophone})} \textcolor{red}{(M7)}}
\begin{description}
  \item[Secrets:] Method of converting analog signal to digital signal, and relaying this signal to the microcontroller processor.
  \item[Services:] Provides microcontroller with some of the input data.
  \item[Implemented By:] Arduino Library
  \end{description}

\section{Traceability Matrix} \label{SecTM}

This section shows two traceability matrices: between the modules and the
requirements and between the modules and the anticipated changes.

% the table should use mref, the requirements should be named, use something
% like fref
\begin{table}[H]
\centering
\begin{tabular}{p{0.2\textwidth} p{0.6\textwidth}}
\toprule
\textbf{Req.} & \textbf{Modules \textcolor{red}{(M2, M6 - Design Decision Removal; M7 - Out of Scope)}}\\
\midrule

R1 & \sout{(\mref{mMicrophone})} \textcolor{red}{(M7)}\\
R2 & \sout{(\mref{mSound})} \textcolor{red}{(M5)}\\
R3 & \sout{\mref{mLogin},} \sout{(\mref{mBluetooth})} \textcolor{red}{(M2)}, \sout{(\mref{mKeyword})} \textcolor{red}{(M3)}, \sout{(\mref{mComm})} \textcolor{red}{(M6)} \\
R4 & \sout{(\mref{mOutput})} \textcolor{red}{(M4)}\\
R5 & \sout{(\mref{mSound})} \textcolor{red}{(M5)}\\
R6 & \sout{(\mref{mBluetooth})} \textcolor{red}{(M2)}\\
\sout{R7} &  \sout{\mref{mLogin}}, \sout{\mref{mProfile}}\\
R8 & \sout{(\mref{mBluetooth})} \textcolor{red}{(M2)}\sout{, \mref{mBattery}}\\
\bottomrule
\end{tabular}
\caption{Trace Between Requirements and Modules}
\label{TblRT}
\end{table}

\begin{table}[H]
\centering
\begin{tabular}{p{0.2\textwidth} p{0.6\textwidth}}
\toprule
\textbf{AC} & \textbf{Modules \textcolor{red}{(M2, M6 - Design Decision Removal; M7 - Out of Scope)}}\\
\midrule
\acref{ac1} & \sout{(\mref{mSound})} \textcolor{red}{(M5)}, \sout{(\mref{mMicrophone})} \textcolor{red}{(M7)}\\
\acref{ac2} & \sout{(\mref{mSound})} \textcolor{red}{(M5)}\\
\acref{ac3} & \sout{\mref{mLogin}}, \sout{(\mref{mBluetooth})} \textcolor{red}{(M2)}, \sout{(\mref{mKeyword})} \textcolor{red}{(M3)}, \sout{\mref{mProfile}}, \sout{\mref{mBattery}}\\
\sout{\acref{ac4}} & \sout{\mref{mBattery}}\\
\acref{ac5} & \sout{(\mref{mBluetooth})} \textcolor{red}{(M2)}, \sout{(\mref{mComm})} \textcolor{red}{(M6)}\\
\acref{ac6} & \sout{(\mref{mSound})} \textcolor{red}{(M5)}, \sout{(\mref{mMicrophone})} \textcolor{red}{(M7)}\\
\sout{\acref{ac7}} & \sout{\mref{mLogin}}\\
\bottomrule
\end{tabular}
\caption{Trace Between Anticipated Changes and Modules}
\label{TblACT}
\end{table}

\section{Use Hierarchy Between Modules} \label{SecUse}

In this section, the uses hierarchy between modules is
provided. \citet{Parnas1978} said of two programs A and B that A {\em uses} B if
correct execution of B may be necessary for A to complete the task described in
its specification. That is, A {\em uses} B if there exist situations in which
the correct functioning of A depends upon the availability of a correct
implementation of B.  Figure \ref{FigUH} illustrates the use relation between
the modules. It can be seen that the graph is a directed acyclic graph
(DAG). Each level of the hierarchy offers a testable and usable subset of the
system, and modules in the higher level of the hierarchy are essentially simpler
because they use modules from the lower levels.

\section{Module Interface Specification Tetrileet}



\begin{figure}[H]
\centering
\includegraphics[width=0.7\textwidth]{UsesHierarchy.png}
\caption{Use hierarchy among modules}
\label{FigUH}
\end{figure}

%\section*{References}



\bibliographystyle {plainnat}
\bibliography{../../../refs/References}

\newpage{}

\end{document}