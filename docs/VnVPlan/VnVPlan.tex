\documentclass[12pt, titlepage]{article}

\usepackage{booktabs}
\usepackage{tabularx}
\usepackage{hyperref}
\hypersetup{
    colorlinks,
    citecolor=blue,
    filecolor=black,
    linkcolor=red,
    urlcolor=blue
}
\usepackage[round]{natbib}
\usepackage{enumitem}

%% Comments

\usepackage{color}

\newif\ifcomments\commentstrue %displays comments
%\newif\ifcomments\commentsfalse %so that comments do not display

\ifcomments
\newcommand{\authornote}[3]{\textcolor{#1}{[#3 ---#2]}}
\newcommand{\todo}[1]{\textcolor{red}{[TODO: #1]}}
\else
\newcommand{\authornote}[3]{}
\newcommand{\todo}[1]{}
\fi

\newcommand{\wss}[1]{\authornote{blue}{SS}{#1}} 
\newcommand{\plt}[1]{\authornote{magenta}{TPLT}{#1}} %For explanation of the template
\newcommand{\an}[1]{\authornote{cyan}{Author}{#1}}

%% Common Parts

\newcommand{\progname}{ProgName} % PUT YOUR PROGRAM NAME HERE
\newcommand{\authname}{Team \#, Team Name
\\ Student 1 name
\\ Student 2 name
\\ Student 3 name
\\ Student 4 name} % AUTHOR NAMES                  

\usepackage{hyperref}
    \hypersetup{colorlinks=true, linkcolor=blue, citecolor=blue, filecolor=blue,
                urlcolor=blue, unicode=false}
    \urlstyle{same}
                                


\begin{document}

\title{Synesthesia Wear: System Verification and Validation Plan for \progname{}} 
\author{\authname}
\date{\today}
	
\maketitle

\pagenumbering{roman}

\section{Revision History}

\begin{tabularx}{\textwidth}{p{3cm}p{2cm}X}
\toprule {\bf Date } & {\bf Version} & {\bf Notes}\\
\midrule
10/31/2022 & 1.0 & Added Section 6 - Unit Test Description\\
Date 2 & 1.1 & Notes\\
\bottomrule
\end{tabularx}

\newpage

\tableofcontents

\listoftables
\wss{Remove this section if it isn't needed}

\listoffigures
\wss{Remove this section if it isn't needed}

\newpage

\section{Symbols, Abbreviations and Acronyms}

\renewcommand{\arraystretch}{1.2}
\begin{tabular}{l l} 
  \toprule		
  \textbf{symbol} & \textbf{description}\\
  \midrule 
  T & Test\\
  \bottomrule
\end{tabular}\\

\wss{symbols, abbreviations or acronyms -- you can simply reference the SRS
  \citep{SRS} tables, if appropriate}

\newpage

\pagenumbering{arabic}

This document ... \wss{provide an introductory blurb and roadmap of the
  Verification and Validation plan}

\section{General Information}

\subsection{Summary}

\wss{Say what software is being tested.  Give its name and a brief overview of
  its general functions.}

\subsection{Objectives}

\wss{State what is intended to be accomplished.  The objective will be around
  the qualities that are most important for your project.  You might have
  something like: ``build confidence in the software correctness,''
  ``demonstrate adequate usability.'' etc.  You won't list all of the qualities,
  just those that are most important.}

\subsection{Relevant Documentation}

\wss{Reference relevant documentation.  This will definitely include your SRS
  and your other project documents (MG, MIS, etc).  You can include these even
  before they are written, since by the time the project is done, they will be
  written.}

\citet{SRS}

\section{Plan}

\wss{Introduce this section.   You can provide a roadmap of the sections to
  come.}

\subsection{Verification and Validation Team}

\wss{You, your classmates and the course instructor.  Maybe your supervisor.
  You shoud do more than list names.  You should say what each person's role is
  for the project.  A table is a good way to summarize this information.}

\subsection{SRS Verification Plan}

\wss{List any approaches you intend to use for SRS verification.  This may just
  be ad hoc feedback from reviewers, like your classmates, or you may have
  something more rigorous/systematic in mind..}

\wss{Remember you have an SRS checklist}

\subsection{Design Verification Plan}

\wss{Plans for design verification}

\wss{The review will include reviews by your classmates}

\wss{Remember you have MG and MIS checklists}

\subsection{Implementation Verification Plan}

\wss{You should at least point to the tests listed in this document and the unit
  testing plan.}

\wss{In this section you would also give any details of any plans for static verification of
  the implementation.  Potential techniques include code walkthroughs, code
  inspection, static analyzers, etc.}

\subsection{Automated Testing and Verification Tools}

\wss{What tools are you using for automated testing.  Likely a unit testing
  framework and maybe a profiling tool, like ValGrind.  Other possible tools
  include a static analyzer, make, continuous integration tools, test coverage
  tools, etc.  Explain your plans for summarizing code coverage metrics.
  Linters are another important class of tools.  For the programming language
  you select, you should look at the available linters.  There may also be tools
  that verify that coding standards have been respected, like flake9 for
  Python.}

\wss{The details of this section will likely evolve as you get closer to the
  implementation.}

\subsection{Software Validation Plan}

\wss{If there is any external data that can be used for validation, you should
  point to it here.  If there are no plans for validation, you should state that
  here.}

\section{System Test Description}
	
\subsection{Tests for Functional Requirements}

\wss{Subsets of the tests may be in related, so this section is divided into
  different areas.  If there are no identifiable subsets for the tests, this
  level of document structure can be removed.}

\wss{Include a blurb here to explain why the subsections below
  cover the requirements.  References to the SRS would be good.}

\subsubsection{Area of Testing1}

\wss{It would be nice to have a blurb here to explain why the subsections below
  cover the requirements.  References to the SRS would be good.  If a section
  covers tests for input constraints, you should reference the data constraints
  table in the SRS.}
		
\paragraph{Title for Test}

\begin{enumerate}

\item{test-id1\\}

Control: Manual versus Automatic
					
Initial State: 
					
Input: 
					
Output: \wss{The expected result for the given inputs}

Test Case Derivation: \wss{Justify the expected value given in the Output field}
					
How test will be performed: 
					
\item{test-id2\\}

Control: Manual versus Automatic
					
Initial State: 
					
Input: 
					
Output: \wss{The expected result for the given inputs}

Test Case Derivation: \wss{Justify the expected value given in the Output field}

How test will be performed: 

\end{enumerate}

\subsubsection{Area of Testing2}

...

\subsection{Tests for Nonfunctional Requirements}

\wss{The nonfunctional requirements for accuracy will likely just reference the
  appropriate functional tests from above.  The test cases should mention
  reporting the relative error for these tests.}

\wss{Tests related to usability could include conducting a usability test and
  survey.}

\subsubsection{Area of Testing1}
		
\paragraph{Title for Test}

\begin{enumerate}

\item{test-id1\\}

Type: 
					
Initial State: 
					
Input/Condition: 
					
Output/Result: 
					
How test will be performed: 
					
\item{test-id2\\}

Type: Functional, Dynamic, Manual, Static etc.
					
Initial State: 
					
Input: 
					
Output: 
					
How test will be performed: 

\end{enumerate}

\subsubsection{Area of Testing2}

...

\subsection{Traceability Between Test Cases and Requirements}

\wss{Provide a table that shows which test cases are supporting which
  requirements.}

\section{Unit Test Description}

\wss{Reference your MIS and explain your overall philosophy for test case
  selection.}  
\wss{This section should not be filled in until after the MIS has
  been completed.}

\subsection{Unit Testing Scope}
The scope of the unit testing will involve evaluating the microphone, bluetooth, 
classification, feedback, noise filter, and interface modules to see if they adhere 
to respective functional and non-functional requirements found in Synesthesia Wear’s 
SRS document.

\subsection{Tests for Functional Requirements}

\wss{Most of the verification will be through automated unit testing.  If
  appropriate specific modules can be verified by a non-testing based
  technique.  That can also be documented in this section.}

\subsubsection{Microphone Module}

\wss{Include a blurb here to explain why the subsections below cover the module.
  References to the MIS would be good.  You will want tests from a black box
  perspective and from a white box perspective.  Explain to the reader how the
  tests were selected.}

\begin{enumerate}

\item{test-id1\\}

\begin{tabular}{ |p{5cm}||p{7cm}| }
    \hline
    Type & Functional, Dynamic, and Manual. \\
    \hline
    Initial State  &  No data in buffer and requesting microphone input.\\
    \hline
    Input &   Sample Recording.  \\
    \hline
    Output &   The sample recording in the memory buffer.  \\
    \hline
    Test Case Derivation &   The output has to be the digital representation of the input.\\
    \hline
    How test will be performed & 3 Different sample sounds will be supplied near the microphone. Will compare the output with expected output. The test succeeds if all the outputs match the expected outputs within some tolerance.     \\
    \hline
\end{tabular}
    
\end{enumerate}

\subsubsection{Bluetooth Module}

\begin{enumerate}

\item{test-id1\\}

\begin{tabular}{ |p{5cm}||p{7cm}| }
    \hline
    Type & Functional, Dynamic, and Manual. \\
    \hline
    Initial State  &  Data in buffer and send request received. \\
    \hline
    Input &   Digital sound recording.  \\
    \hline
    Output &   The same digital sound recording at the receiver.  \\
    \hline
    Test Case Derivation &   The module is a communication module and no change has been made to the data. Hence the data has to be the same as the output.\\
    \hline
    How test will be performed & A large audio recording will be sent to the data buffer of the sender and send request will be asserted. The receiver should receive the data. The data will be compared manually to check if the test was passed.\\
    \hline
\end{tabular}

\item{test-id2\\}

\begin{tabular}{ |p{5cm}||p{7cm}| }
    \hline
    Type & Functional, Dynamic, and Manual. \\
    \hline
    Initial State  &  Classification detected asserted. \\
    \hline
    Input &   Sample classification signal asserted on software.  \\
    \hline
    Output &   Feedback signal asserted on hardware.  \\
    \hline
    Test Case Derivation &   The module is a communication module, and the classification signal received from the software has to tie into its respective feedback signal.\\
    \hline
    How test will be performed & A classification signal will be asserted manually in the software, its respective feedback signal needs to be asserted in the hardware for the test to pass.\\
    \hline
\end{tabular}

\end{enumerate}


\subsubsection{Classification Module}

\begin{enumerate}

\item{test-id1\\}

\begin{tabular}{ |p{5cm}||p{7cm}| }
    \hline
    Type & Functional, Dynamic, and Automatic. \\
    \hline
    Initial State  &  Sound classification settings already preconfigured.\\
    \hline
    Input &   Stored sound data in the memory buffer.  \\
    \hline
    Output &   Classified sound data.  \\
    \hline
    Test Case Derivation &   The output should be digital sound data that has been classified under one of the categories that were preconfigured in the sound classification settings.\\
    \hline
    How test will be performed & Sound data from the Microphone module testing will be used for this test. The classification code ingrained in the Synesthesia Wear app will automatically try to classify stored sound data in memory. The test succeeds if all outputs are classified under their expected categories.\\
    \hline
\end{tabular}

\item{test-id2\\}

\begin{tabular}{ |p{5cm}||p{7cm}| }
    \hline
    Type & Functional, Dynamic, and Manual. \\
    \hline
    Initial State  &  Sound classification settings are empty or already preconfigured. \\
    \hline
    Input &   New classification settings.  \\
    \hline
    Output &   Classification settings have been changed.  \\
    \hline
    Test Case Derivation &   The output should match the new sound classification settings verbatim.\\
    \hline
    How test will be performed & New sound classification settings will be inputted into a menu on the Synesthesia Wear app and a save button will be used to preserve those settings. The test succeeds if after going back to the sound classification settings menu, the newly inputted settings are displayed.\\
    \hline
\end{tabular}

\end{enumerate}


\subsubsection{Feedback Module}

\begin{enumerate}

\item{test-id1\\}

\begin{tabular}{ |p{5cm}||p{7cm}| }
    \hline
    Type & Functional, Dynamic, and Manual. \\
    \hline
    Initial State  &  Classification received. \\
    \hline
    Input &   A feedback signal is asserted.  \\
    \hline
    Output &   Vibration detected at the end that coincides with the feedback signal.  \\
    \hline
    Test Case Derivation &   Tests how our feedback structure performs. \\
    \hline
    How test will be performed & A feedback signal pertaining to a particular classification is asserted, the output has to be equal to the set vibration specified by the classification. \\
    \hline
\end{tabular}

\end{enumerate}

\subsection{Tests for Nonfunctional Requirements}

\wss{If there is a module that needs to be independently assessed for
  performance, those test cases can go here.  In some projects, planning for
  nonfunctional tests of units will not be that relevant.}

\wss{These tests may involve collecting performance data from previously
  mentioned functional tests.}

\subsubsection{Microphone Module}

\begin{enumerate}

\item{test-id1\\}

\begin{tabular}{ |p{5cm}||p{7cm}| }
    \hline
    Type & Dynamic and Manual. \\
    \hline
    Initial State  &  No data in buffer. \\
    \hline
    Input &   Sample recording. \\
    \hline
    Output &   The sample recording in the memory buffer.  \\
    \hline
    Test Case Derivation &   The output has to be within at least a 95\% confidence level of the input. \\
    \hline
    How test will be performed & 3 different sounds found online will be taken and played on some speakers that will project the sounds into the microphone. Taking the initial sound files and the sound data from the microphone, an online software tool will compare the sound data and measure their similarities/confidence level. The test succeeds if the similarities/confidence level is at least 95\%. \\
    \hline
\end{tabular}

\item{test-id2\\}

\begin{tabular}{ |p{5cm}||p{7cm}| }
    \hline
    Type & Dynamic and Automatic. \\
    \hline
    Initial State  &  No data in buffer and the device is powered on. \\
    \hline
    Input &   Random ambient sound. \\
    \hline
    Output &   Continuously updated sound buffer with sampling frequency fs.  \\
    \hline
    Test Case Derivation &   Tests if the device is able to continuously update when turned on. \\
    \hline
    How test will be performed & Random sounds will be inserted into the microphone. The sound buffer will be copied at the frequency of the sampling frequency into a file. The device has to be able to update the sound buffer continuously until the device is turned off to receive a conditional pass. For a complete pass, all the sound data has to have a distortion of less than 5\%. \\
    \hline
\end{tabular}

\end{enumerate}


\subsubsection{Bluetooth Module}

\begin{enumerate}

\item{test-id1\\}

\begin{tabular}{ |p{5cm}||p{7cm}| }
    \hline
    Type & Dynamic and Manual. \\
    \hline
    Initial State  &  Bluetooth device not paired. \\
    \hline
    Input &   Introduce a new bluetooth connection. \\
    \hline
    Output &   Connect with the bluetooth connection in under a minute.  \\
    \hline
    Test Case Derivation &   The device has to be able to connect with the hardware easily. \\
    \hline
    How test will be performed & A new bluetooth device will be introduced to the hardware, on performing the bluetooth connection procedure the connection should be established within a minute for the test to pass. \\
    \hline
\end{tabular}

\item{test-id2\\}

\begin{tabular}{ |p{5cm}||p{7cm}| }
    \hline
    Type & Dynamic and Manual. \\
    \hline
    Initial State  &  Bluetooth device not connected but paired. \\
    \hline
    Input &   Disconnect bluetooth abruptly. \\
    \hline
    Output &   Auto-reconnection of the bluetooth.  \\
    \hline
    Test Case Derivation &   The device has to be able to reconnect without any issues. \\
    \hline
    How test will be performed & The device will be paired to the hardware initially, by taking the device out of range we will simulate abrupt interruption. It should automatically connect back when back in range, this should not take any longer than 10 seconds after the device is back in range. \\
    \hline
\end{tabular}

\end{enumerate}


\subsubsection{Noise Filter Module}

\begin{enumerate}

\item{test-id1\\}

\begin{tabular}{ |p{5cm}||p{7cm}| }
    \hline
    Type & Dynamic and Automatic. \\
    \hline
    Initial State  &  Is empty and waiting for an input to process. \\
    \hline
    Input &   Digital data with one or more sounds. \\
    \hline
    Output &   The same digital sound recording but with less noise.  \\
    \hline
    Test Case Derivation &   The background noise in the sound file is reduced/removed and a main/singular sound is more notable than others. \\
    \hline
    How test will be performed & After receiving sound data over bluetooth, Synesthesia Wear’s app will automatically send this data over to the corresponding device’s noise filtering hardware that will process and return a filtered version of the data. This test passes if it is clear that there is notably less noise in the filtered sound file compared to the original one. \\
    \hline
\end{tabular}

\end{enumerate}


\subsubsection{Classification Module}

\begin{enumerate}

\item{test-id1\\}

\begin{tabular}{ |p{5cm}||p{7cm}| }
    \hline
    Type & Dynamic and Automatic. \\
    \hline
    Initial State  &  Waiting for sound input and classification settings to be preconfigured. \\
    \hline
    Input &   Sample sounds that fall into classifications and those that do not. \\
    \hline
    Output &   Classification signals asserted for sounds that are in the classification.  \\
    \hline
    Test Case Derivation &   Tests the performance and effectiveness of the classification module to be able to distinguish classified and non-classified signals. \\
    \hline
    How test will be performed & A sample set of different sounds (6 different types of sounds with each one supplied 20 times, each time with a random distortion added to make them all digitally different) will be run through a pre-configured classification set. If the output of the module is correct 90\% of the time, it is considered to be a pass. \\
    \hline
\end{tabular}

\end{enumerate}


\subsubsection{Feedback Module}

\begin{enumerate}

\item{test-id1\\}

\begin{tabular}{ |p{5cm}||p{7cm}| }
    \hline
    Type & Dynamic and Manual. \\
    \hline
    Initial State  &  Classification received. \\
    \hline
    Input &   A feedback signal is asserted. \\
    \hline
    Output &   Vibration detected at the end that coincides with the feedback signal and is not intrusive.  \\
    \hline
    Test Case Derivation &   Tests how our feedback structure performs. \\
    \hline
    How test will be performed & A feedback signal pertaining to a particular classification is asserted such that the output has to be equal to the set vibration specified by the classification. A sample group of 5 will be asked to feel the vibration and then reply if said vibration was sufficient and non-intrusive. If 4 of the 5 answers are yes, the test is passed. \\
    \hline
\end{tabular}

\end{enumerate}



\subsubsection{Interface Module}

\begin{enumerate}

\item{test-id1\\}

\begin{tabular}{ |p{5cm}||p{7cm}| }
    \hline
    Type & Structural, Dynamic and Manual. \\
    \hline
    Initial State  &  N/A. \\
    \hline
    Input &   Sample group user inputs. \\
    \hline
    Output &   UI should behave as required in terms of appearance and style.  \\
    \hline
    Test Case Derivation &   Gathering information on the wants of the users to ensure that the interface adheres to their preferences and that they are pleased with their user experiences. \\
    \hline
    How test will be performed & A sample group of 5 people will have the opportunity to interact with the UI and device first. They will later be asked questions regarding certain features of the product. These questions are listed in the appendix. If a satisfying result is obtained over the sample group, then the test is passed. This test has multiple subsections where each can be passed or failed separately. \\
    \hline
\end{tabular}

\item{test-id2\\}

\begin{tabular}{ |p{5cm}||p{7cm}| }
    \hline
    Type & Dynamic and Automatic. \\
    \hline
    Initial State  &  User interface opened up. \\
    \hline
    Input &   User input. \\
    \hline
    Output &   UI response within 1ms.  \\
    \hline
    Test Case Derivation &   The device has to be able to respond quickly to any user input. \\
    \hline
    How test will be performed & It will be too hard to measure the response time manually as most humans have a response time greater than 1ms. Hence this test will be done with the help of helper code which will calculate the time between a user input detected and a corresponding change in the UI. \\
    \hline
\end{tabular}

\item{test-id3\\}

\begin{tabular}{ |p{5cm}||p{7cm}| }
    \hline
    Type & Dynamic and Manual. \\
    \hline
    Initial State  &  User interface opened up. \\
    \hline
    Input &   User input. \\
    \hline
    Output &   Expected UI response on all the different devices.  \\
    \hline
    Test Case Derivation &   The UI has to be able to work the same on all platforms. \\
    \hline
    How test will be performed & The UI will be installed on different systems (Android, Windows, IOS). If it is capable of all functionality within all the platforms, it receives a pass. \\
    \hline
\end{tabular}

\end{enumerate}


\subsection{Traceability Between Test Cases and Modules}

\wss{Provide evidence that all of the modules have been considered.}
				
\bibliographystyle{plainnat}

\bibliography{../../refs/References}

\newpage

\section{Appendix}

This is where you can place additional information.

\subsection{Symbolic Parameters}

The definition of the test cases will call for SYMBOLIC\_CONSTANTS.
Their values are defined in this section for easy maintenance.

\subsection{Usability Survey Questions?}

\subsubsection{Appearance Requirements}

\begin{enumerate}[noitemsep, nolistsep]
\item{How did the finish and look of the device appeal to you?\\}
\item{How was the appearance of different pages in the UI software?\\}

\end{enumerate}

\noindent Expected answers for pass condition: Satisfied or better for both questions above.


\subsubsection{Style Requirements}

\begin{enumerate}[noitemsep, nolistsep]
\item{Did you feel that there was consistency between different elements of the UI?\\}

\end{enumerate}

\noindent Expected answers for pass condition: Yes.


\subsubsection{Ease of Use Requirements}

\begin{enumerate}[noitemsep, nolistsep]
\item{Out of 10, how easy do you find it to interact with the UI?\\}
\item{Out of 10, what would you rate the usability of the system?\\}
\item{What do you find most frustrating about the system?\\}

\end{enumerate}

\noindent Expected answers for pass condition: For the first two questions, the average score has to be greater than 7. The last question should not have the same answer repeated between different members. If so, it would suggest an issue with the system. 


\subsubsection{Personalization and Internationalization Requirements}

\begin{enumerate}[noitemsep, nolistsep]
\item{Were you satisfied with the personalization choices of the UI?\\}

\end{enumerate}

\noindent Expected answers for pass condition: Should be yes for 85\% of the sample group. 


\subsubsection{Learning Requirements}

\begin{enumerate}[noitemsep, nolistsep]
\item{How long did it take you to understand and use the software on your own?\\}

\end{enumerate}

\noindent Expected answers for pass condition: Should not be longer than 5 minutes for each person in the sample group. 


\subsubsection{Understandability and Politeness Requirements}

\begin{enumerate}[noitemsep, nolistsep]
\item{How difficult was it to read information off the screen?\\}
\item{Were you satisfied with the arrangement of content on the screen?\\}
\item{Were you displeased with the language or content used on the UI?\\}

\end{enumerate}

\noindent Expected answers for pass condition: For the first condition, the difficulty should not be more than 6 out of 10. For the second condition, it should be yes for 85\% of the sample group. For the third condition, it should be no for all members of the sample group.

\newpage{}
\section*{Appendix --- Reflection}

The information in this section will be used to evaluate the team members on the
graduate attribute of Lifelong Learning.  Please answer the following questions:

This section deals with what knowledge and experiences each team member will need 
to acquire so that the capstone project can be completed successfully. With that in 
mind, before identifying what each member is going to learn/master, some approaches 
of learning need to be established. Firstly, we believe that one of our main approaches 
to learning/mastering new skills is by scouring the internet for resources, videos, websites, 
blogs, or any other notable sources for relevant information. Another approach would be to 
look through books at McMaster's library to see if there is any applicable details that could 
be used for this project. Furthermore, one could also master their new skills via practice 
and trial-and-error by following tutorials and then trying to do them in real-time. Lastly, a 
final approach could be to find someone with relevant expertise and ask them for advice or 
some lessons on relevant skills/knowledge that would be beneficial for the project as a whole.


\begin{enumerate}
  \item 
  \item 
\end{enumerate}

\end{document}