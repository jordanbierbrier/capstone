\documentclass{article}

\usepackage{booktabs}
\usepackage{tabularx}
\usepackage{verbatim}
\title{Development Plan\\\progname}

\author{\authname}

\date{}

%% Comments

\usepackage{color}

\newif\ifcomments\commentstrue %displays comments
%\newif\ifcomments\commentsfalse %so that comments do not display

\ifcomments
\newcommand{\authornote}[3]{\textcolor{#1}{[#3 ---#2]}}
\newcommand{\todo}[1]{\textcolor{red}{[TODO: #1]}}
\else
\newcommand{\authornote}[3]{}
\newcommand{\todo}[1]{}
\fi

\newcommand{\wss}[1]{\authornote{blue}{SS}{#1}} 
\newcommand{\plt}[1]{\authornote{magenta}{TPLT}{#1}} %For explanation of the template
\newcommand{\an}[1]{\authornote{cyan}{Author}{#1}}

%% Common Parts

\newcommand{\progname}{ProgName} % PUT YOUR PROGRAM NAME HERE
\newcommand{\authname}{Team \#, Team Name
\\ Student 1 name
\\ Student 2 name
\\ Student 3 name
\\ Student 4 name} % AUTHOR NAMES                  

\usepackage{hyperref}
    \hypersetup{colorlinks=true, linkcolor=blue, citecolor=blue, filecolor=blue,
                urlcolor=blue, unicode=false}
    \urlstyle{same}
                                


\begin{document}

\begin{table}[hp]
\caption{Revision History} \label{TblRevisionHistory}
\begin{tabularx}{\textwidth}{llX}
\toprule
\textbf{Date} & \textbf{Developer(s)} & \textbf{Change}\\
\midrule
26.09.2022 & Udeep Shah & Inital commit\\
Date2 & Name(s) & Description of changes\\
... & ... & ...\\
\bottomrule
\end{tabularx}
\end{table}

\newpage

\maketitle

\section{Team Meeting Plan}
The team will meet in-person on Wednesdays at 4:30pm on a weekly basis. These meetings will typically be done in person at Thode Library or at the Capstone project room. However, additional meetings or daily check ups will be scheduled on a per need basis depending on the status of the project and upcoming deliverables. Additional meetings may either be held in person or virtually through the use of the team’s Discord server.

\section{Team Communication Plan}
The team will have two primary sources of communication which include a Discord server as well as a Facebook messenger group chat. The Discord server is used for hosting virtual meetings as well as sharing files and useful links. Communication will also occur through commit messages and issues when collaborating on deliverables through Github. The purpose of using Github will also be for team collaboration as well as changing history and reverting opportunities. 

\section{Team Member Roles}
% Please add the following required packages to your document preamble:
% \usepackage{graphicx}
\begin{comment}
\begin{table}[]
	\centering
	\begin{tabular}{| m{5em} | m{4cm}| m{5cm} |}
	\hline
	Member						 & Role                                           & Description                                                                                                                                                                                                                \\ \hline
	Abraham Taha                 & Application Developer                          & This position is focused on creating a web application for the end user. This will allow the end user to interface with the hardware and change the sounds to detect.  \\ \hline
	Jordan Bierbrier             & Signal Processing / Embedded Systems Developer &                                                                                                                                                                                                      \\ \hline
	Taranjit Lotey               & Application Developer                          & Communication between application and hardware / backend development to send physical signals to user                                                                                                                      \\ \hline
	Azriel Gingoyon              & Hardware Developer                             & Cost-effective component research, wristband design, motor/microcontroller integration                                                                                                                                     \\ \hline
	Udeep Shah                   & Signal Processing / Embedded Systems Developer & Noise filtering, signal isolation resource optimization and power optimization                                                                                                                      \\ \hline
	\end{tabular}
\end{table}
\end{comment}
\subsection*{Abraham Taha}
\textbf{Application Developer} - This position is focused on creating a web application for the
end user. This will allow the end user to interface with the hardware and change the sounds to
detect.

\subsection*{Jordan Bierbrier}
\textbf{Signal Processing / Embedded Systems Developer} - 

\subsection*{Taranjit Lotey}
\textbf{Application Developer} - Communication between application and hardware / backend development to send physical signals to user                                     

\subsection*{Azriel Gingoyon}
\textbf{Signal Processing / Embedded Systems Developer} - Cost-effective component research, wristband design, motor/microcontroller integration                                                                                                                               

\subsection*{Udeep Shah}
\textbf{Signal Processing / Embedded Systems Developer} - Noise filtering, signal isolation resource optimization and power optimization 


\section{Workflow Plan}

\begin{itemize}
	\item Git will be used for collaborating, sharing, tracking and saving our work (code and documents). We will include a main branch, which will consist of the most updated working code. The main branch will connect sub branches of various hardware components, ideally each hardware component will have their own sub branch which will be connected via code in the development branch.
	
	\item Branches will be used when editing or adding code/ documents to the repository. Branches will follow a naming convention for consistency and coherency. They will be created based on issues, for example, IssueNumber-Description-Branch. As a result, allow easy branch linking. Additionally, comments or questions found within an issue can easily be traced to a particular branch. 
	
	\item When a group member wants to merge a branch, a pull request must be created. Pull requests will consist of a comprehensive description of the incoming changes, and additional comments, such as tests conducted. 
	
	\item We will also be using a pre product branch giving the capability to test functionality of referencing branches within each other before making a merge request to main.
	
	\item Additionally there will be a Gitlab continuous integration (CI) test to see if the incoming changes are acceptable (passing a linter test and unit tests). Furthermore, reviewers will be added to assess the changes and leave any comments which will be documented to review and store thought processes at that current stage of development. Once the CI test passes and the reviewers are satisfied with the changes, the reviewers can accept the changes and the branch can be merged into main. 
	
	\item Track issues within Github (Kanban Board) which will reference the issue to designated branch during the pull request. We can tag issues by priority (low, medium, high), by milestone (hardware, application, firmware, etc) and labels. Issues can be assigned to designated group members, and have reviewers to check up on status and work. 
	
		Summarized Github workflow:
	Create and assign issues using GitHub Issues 
	Pull required changes from main branch
	Create a new branch for revision purposes
	Commit change to branch
	Create a pull request for the new changes
	Add required description for the PR and link
	Have assigned members approve changes
		  -     Merge changes after Gitlab continuous integration approves
		  -     Upon approval, remove branch and issue
	
\end{itemize}

\section{Proof of Concept Demonstration Plan}

The major areas where the project could face issues are 
\begin{itemize}
	\item \textbf{Sound recognition} - We might not be able to isolate or classify signals accurately. This is especially hard when there is a lot of background noise. 
	\item \textbf{Vibration motor} - There is a risk of the motor not being powerful enough to get the user’s attention. This may seem easy to tackle by increasing the size and power of the motor but this issues a new issue of power consumption and size increase.
	\item \textbf{Integration of hardware onto a small footprint} - This is the biggest risk with our project. We might not be able to fit all the components i.e the microcontroller, motor, microphone and battery in the small footprint that we desire. The reason why this is hard to tackle is that the smaller we go with our hardware, especially microcontroller and battery, the more cautious we have to be with power consumption and resource allocation.
	The risks associated with our project may change as our development is still ongoing. 
\end{itemize} 

The risks associated with our project may change as our development is still ongoing. 
Our POC will try to focus on the integration of major features into a small footprint (still bigger than the final prototype). The main features of the POC will be sound recognition for three different sounds, vibration alerts and app communication. The successful  demonstration of the POC will certainly reduce the likelihood of not being able to deliver on a final product. The risks that will be difficult to mitigate is the risk of being able to compress our functioning hardware into a small footprint. The main issue here is the size of batteries and the required power consumption for our hardware. A big battery will certainly increase the footprint of our device. At the same time it also has to be sufficient to supply the power that the hardware requires.  


\section{Technology}

\begin{itemize}
\item Specific programming language
\item Specific linter tool (if appropriate)
\item Specific unit testing framework
\item Investigation of code coverage measuring tools
\item Specific plans for Continuous Integration (CI), or an explanation that CI
  is not being done
\item Specific performance measuring tools (like Valgrind), if
  appropriate
\item Libraries you will likely be using?
\item Tools you will likely be using?
\end{itemize}

\section{Coding Standard}
\begin{itemize}
	\item Leave concise and appropriate comments
	\item Name variables in Camel case
	\item Implement exception handling wherever possible
	\item Do not use the same identifier for multiple purposes
	\item Use proper indentation
	\item Conform similar pieces of code into a function
	\item Test the code whenever possible
	\item Avoid lengthy lines of code
	\item Abstain from deep nesting (nesting 4+ levels of code in the same block)
\end{itemize}
\section{Project Scheduling}


\end{document}