\documentclass{article}

\usepackage{booktabs}
\usepackage{tabularx}
\usepackage{hyperref}

\hypersetup{
    colorlinks=true,       % false: boxed links; true: colored links
    linkcolor=red,          % color of internal links (change box color with linkbordercolor)
    citecolor=green,        % color of links to bibliography
    filecolor=magenta,      % color of file links
    urlcolor=cyan           % color of external links
}

\title{Hazard Analysis\\\progname}

\author{\authname}

\date{}

\input{../Comments}
%% Common Parts

\newcommand{\progname}{SE 4G06, TRON 4TB6}
\newcommand{\authname}{Team 26, STRONE 
\\ Jordan Bierbrier
\\ Azriel Gingoyon
\\ Taranjit Lotey
\\ Udeep Shah
\\ Abraham Taha
}                  

\usepackage{hyperref}
    \hypersetup{colorlinks=true, linkcolor=blue, citecolor=blue, filecolor=blue,
                urlcolor=blue, unicode=false}
    \urlstyle{same}
                                


\begin{document}

\maketitle
\thispagestyle{empty}

~\newpage

\pagenumbering{roman}

\begin{table}[hp]
\caption{Revision History} \label{TblRevisionHistory}
\begin{tabularx}{\textwidth}{llX}
\toprule
\textbf{Date} & \textbf{Developer(s)} & \textbf{Change}\\
\midrule
10/14/22 & Azriel G. & Added sections 1, 2, and 3\\
Date2 & Name(s) & Description of changes\\
... & ... & ...\\
\bottomrule
\end{tabularx}
\end{table}

~\newpage

\tableofcontents

~\newpage

\pagenumbering{arabic}

\section{Introduction}
This document is the hazard analysis for the entirety of Synesthesia Wear.
For context, Synesthesia Wear is an inexpensive and non-intrusive hearing aid 
bracelet with a purpose of improving quality of life by providing users with 
an alternate channel for sound recognition within their surroundings. Furthermore, 
this bracelet will have a corresponding application that will be made to be 
user-friendly so that users can easily access and configure their bracelets to 
whatever settings they so desire. Lastly, for the purposes of this document, the 
Synesthesia Wear developers believe that the definition of a hazard is one that is 
derived from Nancy Leveson's work. With that in mind, a hazard is any property or 
condition within the Synesthesia Wear system where after pairing up with any 
conditions in the environment, a potential for loss to the system now exists.

\section{Scope and Purpose of Hazard Analysis}
The scope of this document is to identify any and all possible hazards within 
the system, clarify the mitigation steps of each identified hazard, determine 
the causes and effects of all failures, and define all safety and security requirements 
that have resulted from the overall analysis.

\section{System Boundaries and Components}
The hazard analysis will be conducted on the Synesthesia Wear system which will be 
comprised of the following components:
\begin{enumerate}
    \item The Bracelet which also consists of:
    \begin{enumerate}
        \item Vibration Motor
        \item Sound Sensor
        \item Microcontroller
    \end{enumerate}
    \item The Application to be installed on the users' devices which consists of:
    \begin{enumerate}
        \item User Interface
        \item Bracelet Settings Configuration
    \end{enumerate}
    \item The Device that runs the application
    \begin{enumerate}
        \item Operating System
    \end{enumerate}
\end{enumerate}

With the above in mind, the system boundary is limited to the above 3 components with 
each having their own respective subcomponents. Furthermore, it is important to note 
that not all components in the above list can be controlled (i.e. Device's Operating 
System) by the Synesthesia Wear developers. However, these components still needed to 
be listed down in the system boundary as the potential for a hazard can still be 
correlated to them.

\section{Critical Assumptions}

\wss{These assumptions that are made about the software or system.  You should
minimize the number of assumptions that remove potential hazards.  For instance,
you could assume a part will never fail, but it is generally better to include
this potential failure mode.}

\section{Failure Mode and Effect Analysis}

\wss{Include your FMEA table here}

\section{Safety and Security Requirements}

\wss{Newly discovered requirements.  These should also be added to the SRS.  (A
rationale design process how and why to fake it.)}

\section{Roadmap}

\wss{Which safety requirements will be implemented as part of the capstone timeline?
Which requirements will be implemented in the future?}

\end{document}