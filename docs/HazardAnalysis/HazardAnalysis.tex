\documentclass{article}

\usepackage{booktabs}
\usepackage{tabularx}
\usepackage{hyperref}
\usepackage{pdflscape}
\usepackage[longtable]{multirow}
\usepackage{float}
\usepackage{longtable}
\usepackage{vcell}
\usepackage{pdflscape}
\usepackage{ulem}
\usepackage[margin=1in]{geometry}
\hypersetup{
    colorlinks=true,       % false: boxed links; true: colored links
    linkcolor=red,          % color of internal links (change box color with linkbordercolor)
    citecolor=green,        % color of links to bibliography
    filecolor=magenta,      % color of file links
    urlcolor=cyan           % color of external links
}

\title{Hazard Analysis\\\progname}

\author{\authname}

\date{}

%% Comments

\usepackage{color}

\newif\ifcomments\commentstrue %displays comments
%\newif\ifcomments\commentsfalse %so that comments do not display

\ifcomments
\newcommand{\authornote}[3]{\textcolor{#1}{[#3 ---#2]}}
\newcommand{\todo}[1]{\textcolor{red}{[TODO: #1]}}
\else
\newcommand{\authornote}[3]{}
\newcommand{\todo}[1]{}
\fi

\newcommand{\wss}[1]{\authornote{blue}{SS}{#1}} 
\newcommand{\plt}[1]{\authornote{magenta}{TPLT}{#1}} %For explanation of the template
\newcommand{\an}[1]{\authornote{cyan}{Author}{#1}}

%% Common Parts

\newcommand{\progname}{ProgName} % PUT YOUR PROGRAM NAME HERE
\newcommand{\authname}{Team \#, Team Name
\\ Student 1 name
\\ Student 2 name
\\ Student 3 name
\\ Student 4 name} % AUTHOR NAMES                  

\usepackage{hyperref}
    \hypersetup{colorlinks=true, linkcolor=blue, citecolor=blue, filecolor=blue,
                urlcolor=blue, unicode=false}
    \urlstyle{same}
                                


\begin{document}

\maketitle
\thispagestyle{empty}

~\newpage

\pagenumbering{roman}
\section*{Revision History}
\begin{tabularx}{\textwidth}{llX}
\toprule
\textbf{Date} & \textbf{Developer(s)} & \textbf{Change}\\
\midrule
10/19/22 & Jordan, Abdallah, Udeep & Worked on Sections 5 / General Editing\\
10/14/22 & Azriel G. & Added sections 1, 2, and 3\\
04/04/23 & Taranjit. & Updated Feedback And Formated Table\\
... & ... & ...\\
\bottomrule
	\end{tabularx}

~\newpage

\tableofcontents

~\newpage

\pagenumbering{arabic}

\section{Introduction}
This document is the hazard analysis for the entirety of Synesthesia Wear.
For context, Synesthesia Wear is an inexpensive and non-intrusive hearing aid 
wearable device with a purpose of improving quality of life by providing users with 
an alternate channel for sound recognition within their surroundings. Furthermore, 
this wearable device will have a corresponding application that will be made to be 
user-friendly so that users can easily access and configure their wearable devices to 
whatever settings they so desire. Lastly, for the purposes of this document, the 
Synesthesia Wear developers believe that the definition of a hazard is one that is 
derived from Nancy Leveson's work. With that in mind, a hazard is any property or 
condition within the Synesthesia Wear system where after pairing up with any 
conditions in the environment, a potential for loss to the system now exists.

\section{Scope and Purpose of Hazard Analysis}
The scope of this document is to identify any and all possible hazards within 
the system, clarify the mitigation steps of each identified hazard, determine 
the causes and effects of all failures, and define all safety and security requirements 
that have resulted from the overall analysis.

\section{System Boundaries and Components}
The hazard analysis will be conducted on the Synesthesia Wear system which will be 
comprised of the following components:
\begin{enumerate}
    \item The wearable device consists of:
    \begin{enumerate}
        \item Vibration Motor
        \item Microphone
        \item Microcontroller
        \item Bluetooth module
    \end{enumerate}
    \item The Application to be installed on the users' devices consists of:
    \begin{enumerate}
        \item User Interface
        \item Wearable device Settings Configuration
    \end{enumerate}
    \item The Device that runs the application
    \begin{enumerate}
        \item Operating System
    \end{enumerate}
\end{enumerate}

\noindent With the above in mind, the system boundary is limited to the above 3 components with 
each having their own respective subcomponents. Furthermore, it is important to note 
that not all components in the above list can be controlled (i.e. Device's Operating 
System) by the Synesthesia Wear developers. However, these components still needed to 
be listed down in the system boundary as the potential for a hazard can still be 
correlated to them.

\section{Critical Assumptions}

CA1. The battery will not need to be replaced during product lifespan.
\\CA2. Signal input devices will be consistent with the results they produce.
\\CA3. Software application failure will not diminish usage of product.
\\CA4. The microphone is not blocked and has full access to the environment. 

\section{Failure Mode and Effect Analysis}
\textcolor{red}{Put the table in landscape mode}
%\hspace*{-1cm}
\begin{landscape}
\begin{table}[H]
\centering
    \caption{Wearable Device Failure Modes and Effects Analysis}
	
    %\hspace*{-2.2cm}
    \begin{tabular}{| p{0.1\textwidth} | p{0.08\textwidth}  | p{0.12\textwidth} | p{0.145\textwidth} | p{0.17\textwidth} | p{0.19\textwidth} | p{0.1\textwidth} | p{0.07\textwidth} | p{0.05\textwidth} |}
    \hline
    
    \multicolumn{9}{|l|}{System: Wearable Device} \\
    \multicolumn{9}{|l|}{Phase/Mode: System Requirements} \\ \hline
    \textbf{Sub-System} & \textbf{Design Function} & \textbf{Failure Mode} & \textbf{Effects of Failure} & \textbf{Causes of Failure} & \textbf{Recommended Actions} & \textbf{Risk Priority Number (RPN)} & \textbf{Safety Requirement} & \textbf{Ref} \\ \hline

    Battery & Power the various components of the device  & Battery stops delivering power to the device & 1.Device loses all functionalities & 1.Battery was not charged \newline 2.Battery fails and stops holding charge \newline 3.Battery gets disconnected from the controller & 1.Inform users of best charging practices to avoid battery failure i.e (only charge to 80\%, don't leave it plugged in when battery is full etc.) \newline 2.Microcontroller should throw error code if it detects battery disconnection \newline 3.Have CMOS battery in the Micocontroller incase of power loss & Severity: 10 \newline Occurrence Likelihood: 3 \newline Detection Likelihood: 1 \newline Total: 30 & SIR4, SIR2 & H1-1 \\ \cline{3-9}
    
     & & Battery supplies incorrect power & 1.Devices may lose some functionality or may work incorrectly. The internal components may get damaged & 1.Battery Failure \newline 2.Low charge in the battery \newline 3.Issue in the battery management system (BMS) & 1.Hardware should be able to cut off the battery in case of excess current draw \newline 2.Microcontroller can signal the user in case of low battery & Total: 32 & SIR2 & H1-2 \\ \cline{1-9}

    \end{tabular}
    \hspace*{-1cm}
\end{table}
    
\begin{table}[H]
    \centering
        %\caption{Wearable Device Failure Modes and Effects Analysis}
        
        %\hspace*{-2.2cm}
        \begin{tabular}{| p{0.1\textwidth} | p{0.08\textwidth}  | p{0.12\textwidth} | p{0.145\textwidth} | p{0.17\textwidth} | p{0.19\textwidth} | p{0.1\textwidth} | p{0.07\textwidth} | p{0.05\textwidth} |}
        \hline
        
        \multicolumn{9}{|l|}{System: Wearable Device} \\
        \multicolumn{9}{|l|}{Phase/Mode: System Requirements} \\ \hline
        \textbf{Sub-System} & \textbf{Design Function} & \textbf{Failure Mode} & \textbf{Effects of Failure} & \textbf{Causes of Failure} & \textbf{Recommended Actions} & \textbf{Risk Priority Number (RPN)} & \textbf{Safety Requirement} & \textbf{Ref} \\ \hline
    
        Battery & Power the various components of the device  &  Battery overheats & 1.Device container can melt \newline 2.Battery can melt other components of wearable device \newline 3.Burn the user \newline 4.Damage future battery performance & 1.Device operates in temperatures outside the operating conditions of the battery \newline 2.Battery failure \newline 3.Excessive current draw \newline 4.Loose connections  & 1.Insure proper cooling or heat dissipation of the microcontroller \newline 2.Refer to H1-2 a \newline 3.Install a battery that can operate in the working conditions of the device \newline 4.refer to H1-1 b. & Total: 40 & SIR3 & H1-3 \\ \hline
    
         Microphone & Sound detection & Sound is not detected & 1.\sout{Device is not able to perform the primary function}\textcolor{red}{Device is unable to recognize sounds}  & 1.Loose connections \newline 2.Microphone is damaged  & 1.Microcontroller can throw an error code in case of microphone disconnect \newline 2.User can check the microphone output on the app to see if it is functioning correctly & Total: 30  & IR6  & H2-1 \\ \cline{3-9}
    
         & & Sound is falsely detected & 1.Device functions incorrectly & 1.Loose connections \newline 2.Microphone is damaged & 1.Refer to H2-1.b & Total: 80 & IR6 & H2-2 \\ \hline
     
        \end{tabular}
        \hspace*{-1cm}
\end{table}
\begin{table}[H]
    \centering
        %\caption{Wearable Device Failure Modes and Effects Analysis}
        
        %\hspace*{-2.2cm}
        \begin{tabular}{| p{0.1\textwidth} | p{0.08\textwidth}  | p{0.12\textwidth} | p{0.145\textwidth} | p{0.17\textwidth} | p{0.19\textwidth} | p{0.1\textwidth} | p{0.07\textwidth} | p{0.05\textwidth} |}
        \hline
        
        \multicolumn{9}{|l|}{System: Wearable Device} \\
        \multicolumn{9}{|l|}{Phase/Mode: System Requirements} \\ \hline
        \textbf{Sub-System} & \textbf{Design Function} & \textbf{Failure Mode} & \textbf{Effects of Failure} & \textbf{Causes of Failure} & \textbf{Recommended Actions} & \textbf{Risk Priority Number (RPN)} & \textbf{Safety Requirement} & \textbf{Ref} \\ \hline
    
         Bluetooth Module & Provide a communication stream between mobile phone and wearable device & Mobile device loses connection with  bluetooth module & 1.Sound processing capabilities are lost \newline 2.Vibration motor wont receive signal to provide/not provide haptic feedback  & 1.Signal between mobile phone and  device is lost due to higher than rated distances \newline 2.Signal is blocked due to external factors such as a faraday cage \newline 3.Other signals such as wifi, microwave etc. cause interference with bluetooth signal \newline 4.Connected phone loses power  & 1.Provide a notification to the user when the signal strength is diminished \newline 2.Include auto-reconnection with the device and phone when signal is found \newline 3.Ensure final design of the product has adequate clearing for the bluetooth antennas such that it maximizes signal strength  &  Total: 20 & NFR-8 & H3-1 \\ \cline{3-9}
    
         & & Invalid message & 1.Unexpected or incorrect output from device & 1.Message corrupted during transmission \newline 2.Message corrupted during reception & 1.Add a checksum into the bluetooth signal to check for message integrity \newline 2.Only accept predefined messages, discard foreign/ undefined messages & Total: 15 & IR7 & H3-2 \\ \hline
    
        \end{tabular}
        \hspace*{-1cm}
\end{table}

\begin{table}[H]
    \centering
        %\caption{Wearable Device Failure Modes and Effects Analysis}
        
        %\hspace*{-2.2cm}
        \begin{tabular}{| p{0.1\textwidth} | p{0.08\textwidth}  | p{0.12\textwidth} | p{0.145\textwidth} | p{0.17\textwidth} | p{0.19\textwidth} | p{0.1\textwidth} | p{0.07\textwidth} | p{0.05\textwidth} |}
        \hline
        
        \multicolumn{9}{|l|}{System: Wearable Device} \\
        \multicolumn{9}{|l|}{Phase/Mode: System Requirements} \\ \hline
        \textbf{Sub-System} & \textbf{Design Function} & \textbf{Failure Mode} & \textbf{Effects of Failure} & \textbf{Causes of Failure} & \textbf{Recommended Actions} & \textbf{Risk Priority Number (RPN)} & \textbf{Safety Requirement} & \textbf{Ref} \\ \hline

         Vibration Motor & Provide haptic notification to user & Vibrations not noticeable by user & 1.User does not get alerted & 1.Not enough power supplied & 1.User can calibrate the intensity of the motor & Total: 7 & ACR1 & H4-1 \\ \cline{3-9}
    
         & & Motor does not vibrate & 1.User does not get alerted & 1.Loose connections \newline 2.Defective vibration motor  & 1.Microcontroller can signal the user in case of motor disconnect \newline 2.Refer to  H4-2  & Total: 20 & SIR4 & H4-2 \\ \cline{3-9}
     
         & & Incorrect vibration & 1.User incorrectly identifies the sound & 1.Defective vibration motor & 1.User can calibrate the vibration intensity and check the output  & Total: 18 & SIR4 & H4-3 \\ \cline{3-9}
     
         & & Vibration too intense & 1.Painful or annoying to the user & 1.Motor drawing excess current & 1.Refer to H4-3 \newline 2.Hardware connection is current limited & Total: 8 & ACR1 & H4-4 \\ \hline
     
        \end{tabular}
        \hspace*{-1cm}
\end{table}
%\newpage

\begin{table}[H]

    \caption{Application Failure Modes and Effects Analysis}
    %\scriptsize	
    \centering
    %\hspace*{-2.2cm}
    \begin{tabular}{| p{0.1\textwidth} | p{0.08\textwidth}  | p{0.12\textwidth} | p{0.145\textwidth} | p{0.17\textwidth} | p{0.19\textwidth} | p{0.1\textwidth} | p{0.07\textwidth} | p{0.05\textwidth} |}
     \hline
    
    \multicolumn{9}{|l|}{System: Mobile Application } \\
    \multicolumn{9}{|l|}{Phase/Mode: System Requirements} \\ \hline
    \textbf{Sub-System} & \textbf{Design Function} & \textbf{Failure Mode} & \textbf{Effects of Failure} & \textbf{Causes of Failure} & \textbf{Recommended Actions} & \textbf{Risk Priority Number (RPN)} & \textbf{Safety Requirement} & \textbf{Ref} \\ \hline

    Signal Processing & Classify sound & Sound is incorrectly classified & 1.Incorrectly notify user about sound \newline 2.No notification for detected sound & 1.Insufficient training data \newline 2.Model parameters not fully optimized \newline 3.Outlier sound received  & 1.user can help with calibration by adding more samples \newline 2.Filter outlier noise  & Total: 168 & \sout{ACR2}, IR6 & S1-1 \\ \cline{3-9}
    
     & & Sound is not classified & 1.No notification for detected sound & 1.Error/bug with signal processing code & 1.Refer to S1-1.a & Total: 105 & IR6 & S1-2 \\ \hline

    Graphical User Interface & Give visual representation of the application to the end user & Incompatibility between different mobile devices  & 1.Formatting errors when resizing \newline 2.Unable to download application \newline 3.Loss of functionality or crashing & 1.Button hit box detection may be lost/compromised \newline 2.Mobile OS may not support application \newline 3.Processing power of phone may be too  inadequate for required signal processing \newline 4.Mobile phone may not support bluetooth connections  & 1.Provide end users with a list of certified compatible devices \newline 2.Code/Style the application such that resizing is done automatically as the application detects screen size \newline 3.Update the application on a regular basis to ensure compatibility with latest releases of the OS  & Total: 20 & NFR-7 & S2-1 \\ \cline{1-9}

    \end{tabular}
    %\hspace*{-1cm}
\end{table}

\begin{table}[H]

    %\caption{Application Failure Modes and Effects Analysis}
    %\scriptsize	
    \centering
    %\hspace*{-2.2cm}
    \begin{tabular}{| p{0.1\textwidth} | p{0.08\textwidth}  | p{0.12\textwidth} | p{0.145\textwidth} | p{0.17\textwidth} | p{0.19\textwidth} | p{0.1\textwidth} | p{0.07\textwidth} | p{0.05\textwidth} |}
     \hline
    
    \multicolumn{9}{|l|}{System: Mobile Application } \\
    \multicolumn{9}{|l|}{Phase/Mode: System Requirements} \\ \hline
    \textbf{Sub-System} & \textbf{Design Function} & \textbf{Failure Mode} & \textbf{Effects of Failure} & \textbf{Causes of Failure} & \textbf{Recommended Actions} & \textbf{Risk Priority Number (RPN)} & \textbf{Safety Requirement} & \textbf{Ref} \\ \hline

    Graphical User Interface & Give visual representation of the application to the end user & Combination of user inputs & 1.Loss of saved data \newline 2.Abrupt crashing of the application  &  1.User chooses incorrect bluetooth device to connect to \newline 2.User force closes application before applying changes  &  1.System should recognize invalid inputs from users and provide helpful error messages \newline 2.Application should provide warning when entries are not saved before allowing a force close. Warnings should require user confirmation before allowing the event  & Total: 48 & ACR3 & S2-2 \\ \cline{3-9}

     & & Abnormal closing of application & 1.Loss of saved data \newline 2.Incorrect communication of data  & 1.User closes application while data is being transferred \newline 2.System preemptively forces the application to close & 1.Communication protocol between the device and the application should have error handling in case of errors in data transmission  & Total: 40 & IR3 & S3-3 \\ \hline
    
    \end{tabular}
    %\hspace*{-1cm}
\end{table}

\end{landscape}


%\newpage 
\section{Safety and Security Requirements}

\sout{Bold statements are an extension of the SRS document safety requirements which should have been included in revision 1.}


\subsection{System Isolation Requirements}

SIR1. Product is isolated from electrical components at contact locations. \textcolor{red}{Using wire isolation material}
\\SIR2. Auto shut-off when electrical malfunction detected. 
\\SIR3. Product has sufficient heat dissipation such that all the electrical components stay in their working temperatures.
\\SIR4. System can perform a hard reboot to reset all hardware components.  \textcolor{red}{The built in temperature control system within the Arduino}


\subsection{Access Requirements}

ACR1. Authorized users can access preferred vibration/sensitivity settings through application site.
\sout{\\ACR2. Authorized users can retrain the device through application site.}
\\ACR3. Users given error message if invalid inputs are entered in application.

\subsection{Integrity Requirements}

IR1. Only required variables will be given access to change.
\\IR2. Data will be accessible by authorized users.
\sout{IR3} After synchronization, a copy of data is loaded to system application.
\textcolor{red}{IR3. After synchronization, a copy of the Arduino signal data is loaded to system application.}
\\IR4. No pairs of modes allowed identical settings.
\\IR5. Stored data overridden only at synchronization request.
\\IR6. Application can detect if there is an issue with the microphone.
\\IR7. Unknown messages from Bluetooth module prompt an error to the user and are rejected.
\subsection{Privacy Requirements}

PPR1. Personalized access code will be created for user application accessibility. 
\\PRR2. Data is not transferable between accounts.


\section{Roadmap}

The requirements implemented according to the hardware research milestone created in the development plan are as listed, system isolation requirements and privacy requirements. With access requirements, integrity requirements and privacy requirements being researched in the future development plan.

\end{document}