\documentclass{article}

\usepackage{tabularx}
\usepackage{booktabs}

\title{Problem Statement and Goals\\\progname}

\author{\authname}

\date{}

%% Comments

\usepackage{color}

\newif\ifcomments\commentstrue %displays comments
%\newif\ifcomments\commentsfalse %so that comments do not display

\ifcomments
\newcommand{\authornote}[3]{\textcolor{#1}{[#3 ---#2]}}
\newcommand{\todo}[1]{\textcolor{red}{[TODO: #1]}}
\else
\newcommand{\authornote}[3]{}
\newcommand{\todo}[1]{}
\fi

\newcommand{\wss}[1]{\authornote{blue}{SS}{#1}} 
\newcommand{\plt}[1]{\authornote{magenta}{TPLT}{#1}} %For explanation of the template
\newcommand{\an}[1]{\authornote{cyan}{Author}{#1}}

%% Common Parts

\newcommand{\progname}{ProgName} % PUT YOUR PROGRAM NAME HERE
\newcommand{\authname}{Team \#, Team Name
\\ Student 1 name
\\ Student 2 name
\\ Student 3 name
\\ Student 4 name} % AUTHOR NAMES                  

\usepackage{hyperref}
    \hypersetup{colorlinks=true, linkcolor=blue, citecolor=blue, filecolor=blue,
                urlcolor=blue, unicode=false}
    \urlstyle{same}
                                


\begin{document}

\maketitle

\begin{table}[hp]
\caption{Revision History} \label{TblRevisionHistory}
\begin{tabularx}{\textwidth}{llX}
\toprule
\textbf{Date} & \textbf{Developer(s)} & \textbf{Change}\\
\midrule
September 26th & Abraham Taha & Added Section 1\\
September 26th & Abraham Taha & Added Sections 2,3\\
\bottomrule
\end{tabularx}
\end{table}

\section{Problem Statement}

\subsection{Problem}
In this day and age, communication is a very significant part of our day to day lives. 
So, people who have difficulty in recognizing sounds within their surroundings are not 
able to live their lives to their fullest. For instance, hard of hearing (HoH) individuals 
are not able to always respond promptly whenever their loved ones are calling their name or
when a doorbell rings out for a package delivery. As a result, this can be rather unfortunate
for the HoH individual as inconveniencing loved ones for another appeal or missing a delivery
are both not ideal scenarios to be in. This can especially be the case for life-and-death 
situations where the importance of sound awareness is highlighted. With all this in mind, 
a device which will monitor your surroundings for you would encourage confidence within 
the users to complete daily tasks as well as notably improve their quality of life.
\subsection{Inputs and Outputs}

\subsubsection {Inputs}
\begin{itemize}
    \item Environment Sounds
    \item App Settings/Configurations
\end{itemize}
\subsubsection {Outputs}
\begin{itemize}
    \item Haptic Feedback
\end{itemize}

\subsection{Stakeholders}
\begin{itemize}
    \item Individuals who are hard of hearing, deaf or that are in an environment where hearing can be obstructed
    \item Dr.\ Martin Von Mohrenschildt (Project Supervisor)
    \item STRONE Developers (Creators of the project)
\end{itemize}

\subsection{Environment}
\subsubsection{Hardware}
\begin{itemize}
    \item Device can be configured through the use of a personal computer, laptop or phone.
\end{itemize}
\subsubsection{Software}
\begin{itemize}
    \item The application will be supported on both Mac and Windows operating systems.
    \item The application will be supported on IOS and Andrioid mobile devices.   
\end{itemize}

\section{Goals}
\subsection{Finished device can withstand above moderate
environmental conditions such as moisture and
heat}
\begin{itemize}
    \item The device will be a wearable technology which makes it a necessity that the user does not need to worry about damage in average daily conditions.  
   \end{itemize} 

 \subsection{Device can collect and process data in real time
 while filtering out noise }  
 \begin{itemize}
    \item A core functional aspect of the device is to process a continuous stream of data and react once a key data point is found.
    \item This permits our device to provide the necessary sensory outputs to the user.
  \end{itemize} 

 \subsection{Finished product can communicate with an
 external device}
 \begin{itemize}
    \item Having the ability to directly communicate with external devices will allow the device to be uniquely programmable by different users.
 \end{itemize}

 \subsection{Data collected will be processed and then
 deleted in real time}
 \begin{itemize}
    \item This allows the product to be used in multiple environments without the risk of privacy concerns for the user.
\end{itemize}  

 \subsection{Finished product will retain all functionalities
 in a form factor that is wearable}
 \begin{itemize}
    \item Device will need to be appropriately small such that it can be worn comfortably and non-intrusively by the user.
 \end{itemize}

 \subsection{Finished product will be rechargeable and
 reusable}
 \begin{itemize}
    \item This will expand the usability of the device and help market growth.
    \item This will also contribute to reducing e-waste since the finished product will have a longer lifespan.
 \end{itemize} 

 \subsection{The finished product will have sufficient distinct
 sensory outputs such that the user can
 distinguish between them}
 \begin{itemize}
    \item This allows the device to have multiple programmable key data points that the user can choose to be notified by.
    \item This will also act to expand the usability of the product and programmability of the device by the user. 
 \end{itemize} 

\section{Stretch Goals}

\subsection{Finished product will have a water resistance
rating of IPX5}
\begin{itemize}
    \item This will increase usability of the product.
    \item This will also avoid the possibility that the device will become malfunctional under certain wet weather conditions.
 \end{itemize} 


\subsection{Finished product can actively filter out noise in particularly loud environments}
\begin{itemize}
    \item This will expand the environments where
    the product can function as intended.
    This will also lead to more reliable
    outputs from the device in situations
    where hearing is more obstructed by
    noises in the surroundings.
\end{itemize}

\subsection{Finished product will support over the air firmware updates}
\begin{itemize}
    \item As the product develops and is used in
    real world applications we predict that the
    noise detection algorithm we use will keep
    improving. Having the option to upgrade
    prior devices with better signal processing
    through software will aid in extending the
    usability of older products.
    
\end{itemize}

\end{document}